% !TEX TS-program = pdflatex
% !TEX encoding = UTF-8 Unicode

% This is a simple template for a LaTeX document using the "article" class.
% See "book", "report", "letter" for other types of document.

\documentclass[11pt]{article} % use larger type; default would be 10pt

\usepackage[utf8]{inputenc} % set input encoding (not needed with XeLaTeX)
\usepackage{amsmath}
\usepackage{amsfonts}

\begin{document}
\section{K-Nearest-Neighbour}

Dieses Maschinelle Lernen platziert die Daten in einem $n$-dimensionalen Raum und schaut sich die nächsten $k$-Nachbarn an. Welche Nachbarn das sind entscheidet die Metrik, also eine Abstandsbestimmung in einem Koordinatensystem. Nach Auswahl der nächsten $k$-Nachbarn 
werden die Klassizierungen dieser Nachbarn ausgewertet und durch einen Mehrheitsentscheid entschieden, welche Klassifizierung der
neue Datenpunkt bekommt. \\
Die häufigste Metrik ist die Minkowski-Metrik, da sie für den $n$-dimensionalen euklidischen Raum gilt.

\begin{equation}
D(x,y) = (\sum_{i=1}^n |x_i - y_i|^p )^{\frac{1}{p}}
\end{equation}

$p$ gibt hierbei an, ob es entweder die euklidische oder die Manhattendistanz ist.
\end{document}